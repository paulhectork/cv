\documentclass{prometheus_cv}
\usepackage[utf8]{inputenc}
\usepackage[a4paper, total={5.3in, 10in}, top=2cm]{geometry}		% width=5.3in, height=10in, top margin=2cm on each page

\usepackage[english]{babel}

\usepackage[T1]{fontenc}
\usepackage{csquotes} % french quotes

\usepackage{xcolor}																				  % define some colors
\definecolor{highlight}{HTML}{283559}
\definecolor{highlight2}{HTML}{1A2640}
\definecolor{link}{HTML}{5C6273}

\usepackage{lipsum}
\usepackage{siunitx}							% package to properly set units
\usepackage{fontawesome5}						% package for icons (see list of available icons here: http://mirrors.ibiblio.org/CTAN/fonts/fontawesome5/doc/fontawesome5.pdf)
\usepackage[super]{nth}							% when you write \nth{2} you get a nice superscript
\usepackage[								% coloring of links
	colorlinks = true,
	linkcolor  = link,
	urlcolor   = magenta,
	citecolor  = highlight2
]{hyperref}

\usepackage{fontspec}							% package to change fonts				
\setmainfont[
	BoldFont       = Cormorant Garamond Bold,
	ItalicFont     = Cormorant Garamond Italic,
	BoldItalicFont = Cormorant Garamond Bold Italic
]{Cormorant Garamond Regular}						% Set the main font to Comorant Garamond (https://github.com/CatharsisFonts/Cormorant)
\newfontfamily\GaramondLight{Cormorant Garamond Light}
\newcommand\textlf[1]{{\GaramondLight#1}}

\newcommand{\highlight}[1]{\textcolor{highlight}{\textbf{#1}}}		% highlight text as bold and with the highlight color defined above
\newcommand{\ec}{\textsuperscript{\textdagger}}								% Equal contribution dagger


% define the header and (not) footer %
\usepackage{fancyhdr}																		  
\fancyhf{}
\rhead{Curriculum Vit\ae}
\lhead{Paul Kervegan}
\rfoot{Page \thepage}
\renewcommand{\headrulewidth}{0.5pt}
\renewcommand{\footrulewidth}{0pt}

%%%% biblio
\usepackage[
	backend = biber,
	sorting = ydnt,
	style   = authoryear-ibid
]{biblatex}
\nocite{*} % print all bib entries
\addbibresource{../bib/phk_publications.bib}
\addbibresource{../bib/phk_communications.bib}

% WARNING: The 〈datasource〉 name should be exactly as given in a \addresource macro defining a data source for the document. are allowed within a \map element.
\DeclareSourcemap{%
	\maps[datatype=bibtex]{%
		\map[overwrite]{%
			\perdatasource{../bib/phk_publications.bib}
			\step[fieldset=keywords, fieldvalue={,publications}, append]
		}
		\map[overwrite]{%
			\perdatasource{../bib/phk_communications.bib}
			\step[fieldset=keywords, fieldvalue={,communications}, append]
		}
	}
}



\begin{document}
\thispagestyle{empty}					% Turn off header and footer for the first page
\pagestyle{fancy}			 		% For the rest of the pages switch to the fancy page style defined just above the document begin

%%%%%%%%%%%% TITLE %%%%%%%%%%%% 
\vspace*{-1cm}
\centering 
\begin{scriptsize}
	 \textcolor{black}{Curriculum Vit\ae}
\end{scriptsize}

\vspace*{-0.10em}
\begin{Large} 
	Paul, Hector Kervegan
\end{Large}

\vspace*{0.25em}
\begin{scshape}
	\begin{footnotesize}
		  \textcolor{highlight2}{Humanités numériques $\cdot$ Développement web $\cdot$ Base de données}
		  
		  % \vspace*{-1ex}
		  % \textcolor{highlight2}{Key Word 4 $\cdot$ Key Word 5}
	\end{footnotesize}
\end{scshape}
\vspace*{0.4cm}

\begin{footnotesize}
	\begin{tiny}\faHome\end{tiny}~\href{https://github.com/paulhectork/}{Github}
	\quad \begin{tiny}\faEnvelope[regular]\end{tiny}~\href{mailto:paul.kervegan@proton.me}{%
		paul.kervegan@proton.me
	} 
	
	\begin{tiny}\faMobile*\end{tiny}~\href{tel:001555555555}{
		+33 6 74 40 51 67
	} 
	\quad

\end{footnotesize}

%%%%%%%%%%%% WORK %%%%%%%%%%%% 
\vspace*{0.4cm}
\section{Work experience}
\datedsubsection{ Nov. 2022 -- Dec. 2024 }
	{%
		Institut national d'histoire de l'art, \textit{Richelieu. Histoire du quartier}}
	{%
		\textbf{Research engineer}~}
	{%
		As the research engineer for the \textit{Richelieu. Histoire du quartier} programme, I have been in charge of conceiving and implementing a complete data processing pipeline: full-stack development of a web platform aimed towards researchers (REST API, data visualisation and mapping), PostgreSQL database modelling and administration, data migration, GIS management.}
\datedsubsection{ April 2022 -- Sept. 2022 }
	{%
		École normale supérieure, \textit{MSS/Katabase}}
	{%
		\textbf{End of studies internship}~ École nationale des Chartes}
	{%
		For the \textit{MSS/Katabase} research project, I worked towards automating the management and analysis of a database of 500 XML-TEI documents: text mining, named entity resolution and quantitative analysis. I redesigned of the project's \href{https://katabase.huma-num.fr/}{website} and established a REST API.}
\datedsubsection{ June 2021 -- Sept. 2021 }
	{%
		Archives nationales, \textit{ANR DesignSHS}}
	{%
		\textbf{Internship}~}
	{%
		From the inception of the \textit{DesignSHS} programme (funded by the French National Research Agency, or ANR), I participated in modelling and enriching a database of historical documents, before carrying out an exploratory study in data visualisation and statistical analysis.}
\datedsubsection{ March 2021 -- May 2021 }
	{%
		INA GRM}
	{%
		\textbf{End of studies internship}~École du Louvre}
	{%
		Within the National Audiovisual Institute's Music Reseach Group (INA GRM), I established a database of the GRM's academic publications: archival research, analysis of bibliographic metadata standards, comparative study of open-source database management systems.}


%%%%%%%%%%%% EDUCATION %%%%%%%%%%%% 
\vspace*{0.4cm}
\section{Education}
\datedsubsection{2021 -- 2022}
	{%
		École nationale des Chartes}
	{%
		\textbf{Master's degree}~Digital technologies applied to history}
	{%
		Master's thesis: \textit{Modelling, semanting enrichment and dissemination of semi-structured textual corpora: the case of manuscript sales catalogues}, under the direction of Lucence Ing (École nationale des Chartes). \href{https://dumas.ccsd.cnrs.fr/dumas-04541009v1}{Available online}.}

\datedsubsection{2020 -- 2021}
	{%
		École du Louvre}
	{%
		\textbf{Master's degree}~Documentation and digital humanities}
	{%
		Master's thesis: \textit{From the first processing to open data. Aims and methodology towards a referencing of the INA GRM's academic publications}, under the direction of Françoise Dalex (Musée du Louvre) and Alix Chagué (INRIA).}
\datedsubsection{2019 -- 2020}
	{%
		École du Louvre}
	{%
		\textbf{Master's degree, first year}~Art history and museology}
	{%
		Specialised in \textit{Socio-cultural European anthropology}. First year's master's thesis: \textit{\enquote{Social objects}. The function of instrumentation in noise music}, under the direction of Claire Calogirou (CNRS), Sarah Benhaïm (Université de Tours) and Catherine Guesde (Université Paris 8).}
\datedsubsection{2016 -- 2019}
	{%
		École du Louvre}
	{%
		\textbf{Bachelor degree}~Art history}
	{%
		Double specialisation in \textit{20\textsuperscript{th} century arts} and \textit{Contemporary art}.}
	

%%%%%%%%%%%% COURS %%%%%%%%%%%% 
\section{Teaching and scientific coordination}
\datedsubsection{ 2024 }
	{%
		Institut national d'histoire de l'art}
	{%
		\textbf{Responsable scientifique}~Séminaire \textit{Étudier la ville}}
	{%
		Co-responsable scientifique du séminaire de recherche \textit{Étudier la ville. Un dialogue entre pratique numérique et histoire de l'art}. Élaboration du programme, organisation des séances et invitation des intervenant.e.s.}
\datedsubsection{ 2024 }
	{%
		Conférence Humanistica 2024}
	{%
		\textbf{Membre du comité scientifique}~}
	{%
		Participation au processus de sélection des publications (sélection des intervenant.e.s).}
\datedsubsection{ 11 mars 2024 }
	{%
		École normale supérieure}
	{%
		\textbf{Cours}~Introduction à la fouille de texte}
	{%
		Intervention de 2h lors du \textit{Cours d'introduction aux humanités numériques} organisé par Léa Saint-Raymond: présentation de techniques de \textit{lecture distante} avec Python (\textit{POS-tagging}, densité lexicale). \href{https://github.com/paulhectork/cours_ens2024_fouille_de_texte.git}{Code source en ligne}.}
\datedsubsection{ 30 mars 2024 }
	{%
		École normale supérieure}
	{%
		\textbf{Journée d'atelier}~Modéliser et exploiter des corpus textuels}
	{%
		Co-organisation et animation avec Léa Saint-Raymond d'une journée d'introduction à l'analyse de corpus textuels encodés en XML-TEI: reconnaissance d'entités nommées, géocodage, visualisation cartographique d'un corpus de 267 lettres encodées en XML. \href{https://github.com/paulhectork/cours_ens2023_xmltei}{Code source en ligne}.}
\datedsubsection{ 20 février 2023 }
	{%
		École normale supérieure}
	{%
		\textbf{Cours}~Introduction à la fouille de texte}
	{%	
		Intervention de 2h lors du \textit{Cours d'introduction aux humanités numériques} organisé par Léa Saint-Raymond: présentation de techniques d'analyse de données avec Python (expressions régulières, API, visualisations). \href{https://github.com/paulhectork/cours_ens2024_fouille_de_texte.git}{Code source en ligne}.}



%%%%%%%%%%%% HONORS and SCHOLARSHIPS %%%%%%%%%%%% 
\section{Honours}
\datedsubsection{ 2023 }
	{%
		Ministère de l'enseignement supérieur et de la recherche}
	{%
		\textbf{\enquote{Open science} award}}
	{%
		The \textit{Richelieu. Histoire du quartier} research programme was awarded from the French Ministry of Higher Education an Open science award for research data, in the category \enquote{Reusing data -- young researchers}}
\datedsubsection{ 2020 }
	{%
		IASPM -- BFE}
	{%
		\textbf{\enquote{Young reseachers} award}}
	{%
		My paper \textit{Instrumental transition: redefining the electric guitar through noise practices} was awarded the 2020 young researchers award of the International association for the study of popular music -- Branche Francophone d'Europe.}



%%%%%%%%%%%% PUBLICATIONS %%%%%%%%%%%% 
\printbibliography[title={Publications}, keyword={publications}]
\printbibliography[title={Presentations}, keyword={communications}]

%%%%%%%%%%%% SKILLS %%%%%%%%%%%% 
%\section{Skills}
%Strong knowledge of the programming language \texttt{olymp}\footnotemark with a focus on high-performance generation of Gods.

Working knowledge in \texttt{thor}, \texttt{iliad} (e.g.~\texttt{homer.il}), and $\Theta\text{++}$.

\footnotetext{Example project: \href{https://en.wikipedia.org/wiki/Zeus}{Zeus}}

%%%%%%%%%%%% EXTRAS %%%%%%%%%%%% 
%\section{Extracurricular Activities}
%\datedsubsection{1772 -- 1774}
{Goethe's Book}
{Prometheus}
{Bedecke deinen Himmel, Zeus,
	Mit Wolkendunst!
	Und \"ube, Knaben gleich,
	Der Disteln k\"opft,
	An Eichen dich und Bergesh\"ohn!
	Mußt mir meine Erde
	Doch lassen stehn,
	Und meine H\"utte,
	Die du nicht gebaut,
	Und meinen Herd,
	Um dessen Glut
	Du mich beneidest.
}

\end{document}
