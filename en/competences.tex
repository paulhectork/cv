\datedsubsection{ }
{}
{%
	\textbf{\textit{Full-stack} web development}}
{%
	Development of reactive web applications and communication between several applications using APIs: 
	\begin{itemize}
		\item \textit{Backend}: Backend API and application development using Python-Flask, based on PostgreSQL, SQLite, XML databases and IIIF image corpora.
		\item \textit{Frontend}: HTML / CSS / JavaScript / TypeScript, Vue.js and Angular frameworks, asynchronous communication between frontend and backend, data visualisation (using the JavaScript librairies Leaflet, Plotly, Vis.js and Openseadragon).
	\end{itemize}
}

\datedsubsection{ }
{}
{%
	\textbf{Database management}}
{%
	\begin{itemize}
		\item Relational database administration (PostgreSQL, SQLite).
		\item Integration of databases within reactive applications (Python-SQLAlchemy).
		\item Spatial database management (QGIS, routine automation using Python-QGIS).
		\item Graph database querying (SPARQL, RDF).
	\end{itemize}
}

\datedsubsection{ }
{}
{%
	\textbf{Data analysis and handling}}
{%
	\begin{itemize}
		\item Mass processing, analysis and migration of tabular (Python-Pandas), raster (Python-Pillow) and spatial (GDAL) data.
		\item Natural language processing: named entity recognition and \textit{POS-tagging} using Python (SpaCy, NLTK), statistical analysis of unstructured text, named entity resolution through Wikidata alignment.
		\item Notions in network analysis (Python-NetworkX).
	\end{itemize}
}

\datedsubsection{ }
{}
{%
	\textbf{XML technologies}}
{%
	\begin{itemize}
		\item Knowledge and usage of the XML-TEI and XML-EAD standards.
		\item Automated XML corpora manipulation using Python-LXML, XSLT and BaseX.
	\end{itemize}
}

\datedsubsection{ }
{}
{%
	\textbf{Versionning and deployment}}
{%
	\begin{itemize}
		\item Version control using GIT.
		\item Deployment of Web applications on server: SSH and SCP protocols, notions in Apache server configuration.
	\end{itemize}
}

\datedsubsection{ }
{}
{%
	\textbf{Project management}}
{%
	Responding to call for projects, organization of scientific events, communicating on digital humanities projects to institutional partners and researchers from various academic backgrounds.
}

\datedsubsection{ }
{}
{%
	\textbf{Language proficiency}}
{%
	French (native language), English (bilingual, Cambridge Advanced CAE, CEFR C2 level), Greek (native language), German (CEFR B2 level). }