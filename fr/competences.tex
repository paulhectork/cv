\datedsubsection{ }
{}
{%
	\textbf{Développement applicatif \textit{full-stack}}}
{%
	Développement d'applications réactives et communication entre plusieurs applications via des API: 
	\begin{itemize}
		\item \textit{Backend}: développement d'applications et d'API avec Python-Flask (réponses en HTML et JSON), s'appuyant sur des bases de données PostgreSQL, SQLite, XML et des corpus d'images IIIF.
		\item \textit{Frontend}: HTML / CSS / JavaScript / TypeScript, \textit{frameworks} Vue.js et Angular, communication avec un \textit{backend} Flask via des requêtes asynchrones, visualisation de données (librairies JavaScript Leaflet, Plotly, Vis.js, Openseadragon).
	\end{itemize}
}

\datedsubsection{ }
{}
{%
	\textbf{Gestion de bases de données}}
{%
	\begin{itemize}
		\item Administration de bases de données relationnelles (PostgreSQL, SQLite).
		\item Intégration de bases de données dans des applications réactives (Python-SQLAlchemy).
		\item Gestion de bases de données cartographiques (QGIS, automatisation de routines via des scripts Python-QGIS).
		\item Utilisation de bases de données en graphe (SPARQL, RDF).
	\end{itemize}
}

\datedsubsection{ }
{}
{%
	\textbf{Analyse et manipulation de données}}
{%
	\begin{itemize}
		\item Traitement en masse, analyse et migration de données tabulaires (Python-Pandas), raster (Python-Pillow) et cartographiques (GDAL).
		\item Traitement automatisé du langage: reconnaissance d'entités nommées et \textit{POS-tagging} avec Python (SpaCy et NLTK), analyses statistiques sur du texte brut, résolution d'entités nommées par alignement avec Wikidata.
		\item Notions en analyse de réseaux (Python-NetworkX).
	\end{itemize}
}

\datedsubsection{ }
{}
{%
	\textbf{Technologies XML}}
{%
	\begin{itemize}
		\item Connaissance des standards XML-TEI et XML-EAD.
		\item Manipulation de corpus XML via Python-LXML, XSLT et BaseX.
	\end{itemize}
}

\datedsubsection{ }
{}
{%
	\textbf{Versionnage et déploiement}}
{%
	\begin{itemize}
		\item Outils de versionnage (GIT).
		\item Déploiement d'applications sur Web: protocoles SSH et SCP, notions en paramétrage de serveurs Apache.
	\end{itemize}
}

\datedsubsection{ }
{}
{%
	\textbf{Gestion de projet}}
{%
	Rédaction de réponses à des appels à projets, organisation d'évènements scientifiques, communications autour d'un projet en humanités numériques pour des partenaires institutionnels et des publics de différents niveaux techniques. 
}

\datedsubsection{ }
{}
{%
	\textbf{Compétences linguistiques}}
{%
	Anglais bilingue (Cambridge Advanced CAE, niveau C2), Grec (langue maternelle), Allemand (niveau B2). }