\datedsubsection{ 2024 }
	{%
		Institut national d'histoire de l'art}
	{%
		\textbf{Responsable scientifique}~Séminaire \textit{Étudier la ville}}
	{%
		Co-responsable scientifique du séminaire de recherche \textit{Étudier la ville. Un dialogue entre pratiques numériques et histoire de l'art}. Élaboration du programme, organisation des séances et invitation des intervenant.e.s.}
\datedsubsection{ 2024 }
	{%
		Conférence Humanistica 2024}
	{%
		\textbf{Membre du comité scientifique}~}
	{%
		Participation au processus de sélection et de relecture des publications.}
\datedsubsection{ 11 mars 2024 }
	{%
		École normale supérieure}
	{%
		\textbf{Cours}~Introduction à la fouille de texte (2h)}
	{%
		Intervention lors du \textit{Cours d'introduction aux humanités numériques} organisé par Léa Saint-Raymond: présentation de techniques de lecture distante avec Python (\textit{POS-tagging}, densité lexicale). \href{https://github.com/paulhectork/cours_ens2024_fouille_de_texte.git}{Code source en ligne}.}
\datedsubsection{ 30 mars 2023 }
	{%
		École normale supérieure}
	{%
		\textbf{Journée d'atelier}~Modéliser et exploiter des corpus textuels}
	{%
		Co-organisation et animation avec Léa Saint-Raymond d'une journée d'introduction à l'analyse de corpus textuels encodés en XML-TEI: reconnaissance d'entités nommées, géocodage, visualisation cartographique d'un corpus de 267 lettres. \href{https://github.com/paulhectork/cours_ens2023_xmltei}{Code source en ligne}.}
\datedsubsection{ 20 février 2023 }
	{%
		École normale supérieure}
	{%
		\textbf{Cours}~Introduction à la fouille de texte (2h)}
	{%	
		Intervention lors du \textit{Cours d'introduction aux humanités numériques} organisé par Léa Saint-Raymond: présentation de techniques d'analyse de données avec Python (expressions régulières, API, visualisations). \href{https://github.com/paulhectork/cours_ens2024_fouille_de_texte.git}{Code source en ligne}.}

