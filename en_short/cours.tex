\datedsubsection{ 2024 }
	{%
		Institut national d'histoire de l'art}
	{%
		\textbf{Scientific coordinator}~\textit{Studying the city} seminar}
	{%
		Scientific coordinator of the research seminar: \textit{Studying the city: a dialogue between digital practices and art history}. Conception of the programme, organising the conferences, interlocutor for the guest speakers.}
\datedsubsection{ 2024 }
	{%
		Humanistica 2024 Conference}
	{%
		\textbf{Member of the scientific commitee}~}
	{%
		Participation in the peer review process: evaluation and correction of the submissions.}
\datedsubsection{ 11 march 2024 }
	{%
		École normale supérieure}
	{%
		\textbf{Teaching}~Introduction to text mining (2h)}
	{%
		Part of the \textit{Introduction to Digital humanities} seminar coordinated by Léa Saint-Raymond: hands-on presentation of distant reading techniques using Python (\textit{POS tagging}, lexical density). The source code is \href{https://github.com/paulhectork/cours_ens2024_fouille_de_texte.git}{available online}.}
\datedsubsection{ 30 march 2023 }
	{%
		École normale supérieure}
	{%
		\textbf{Workshop}~Modelling and exploiting textual corpora}
	{%
		Co-organisation and facilitation with Léa Saint-Raymond of a day-long workshop on the analysis of TEI-encoded textual corpora: named entity recognition, geocoding, data visualisation of 267 letters. The presentation and source code are \href{https://github.com/paulhectork/cours_ens2023_xmltei}{available online}.}
\datedsubsection{ 20 february 2023 }
	{%
		École normale supérieure}
	{%
		\textbf{Teaching}~Introduction to text mining (2h)}
	{%	
		Part of the \textit{Introduction to Digital humanities} seminar coordinated by Léa Saint-Raymond. The presentation was focused on data analysis using Python (regular expressions, API, data visualisation). The source code is \href{https://github.com/paulhectork/cours_ens2024_fouille_de_texte.git}{available online}.}

